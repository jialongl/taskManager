\documentclass[12pt, a4paper]{article}
\usepackage{indentfirst}
\usepackage{graphicx}
\graphicspath{{/home/jialongl/Documents/sem3/se/proposal/}{/home/jialongl/Documents/sem3/se/finalprojScreenshots/}}
\usepackage{hyperref}
\hypersetup{
    colorlinks,
% citecolor=black,
% filecolor=black,
% linkcolor=black,
    urlcolor=blue
}
%\usepackage{setspace}
%\onehalfspacing

\begin{document}
\begin{titlepage}
  \centering

  \textsc{\LARGE cs2103 project manual}
  \\[1.5cm]
  \hrule
  \\[0.5cm]
  \Large \bfseries Command line task manager
  \\[0.4cm]
  \hrule
  \\[1.5cm]

  \begin{tabular}{c | c | c | c}
    \includegraphics[scale=0.72]{U099121H}&
    \includegraphics[scale=0.72]{U099122U}&
    \includegraphics[scale=0.72]{U099120W}&
    \includegraphics[scale=0.72]{U094837M}\\

    \small{He Haocong} & \small{Liu Jialong} & \small{Wang Xiangyu} & \small{Zhou Biyan} \\
    \small{\texttt{U099121H}} & \small{\texttt{U099122U}} & \small{\texttt{U099120W}} & \small{\texttt{U094837M}} \\

  \end{tabular}

  \vfill
  {\large \today}
\end{titlepage}


\tableofcontents

\newpage

\section{Introduction}
abc abc

\section{User Guide}
\subsection{Quick Start}

This section introduces you the minimum amount of commands to get you started.

\noindent \begin{enumerate}
\item Start taskManager shell

On Mac OS and GNU Linux (referred to as ``*nix'' later), taskManager can be started from shell by commands:

\texttt{\$ cd the/folder/containing/taskManager\\
  \$ ./taskManager \footnotemark}

On Microsoft Windows, taskManager can be launched in command prompt as well, or simply by double clicking.

Once taskManager is launched, you will see a prompt like ``\textgreater \ \_'', and you will start typing commands!

\footnotetext{ *nix version can be run at any directory after installing taskManger -- ``make install''. See details in section ``Compilation and Installation''.}

\item Add some tasks

\texttt{\$ ./taskManager\\
  \textgreater \  add "Sample task 1"\\
  \textgreater \  add "Sample task 2"\\
  TaskManager: This task is highly similiar to some existing task, do you really want to add it? \textcolor{blue}{y}}

To add a task, simply use the \texttt{add} command followed by the description of the task in a pair of quotation marks. 

\emph{If no error message is shown, the task is successfully added.}
TaskManager may prompt for confirmation if the task to be added is highly similar to some existing task(s) to help prevent people forget adding tasks, which is the case in the example above.

\item List the existing tasks

\texttt{\textgreater \ ls\\
  1   Sample task 1\\
  2   Sample task 2}

To see the existing tasks, use \texttt{ls} command. By default, the taskManager shows the serial number and the description of the tasks.

\item Mark a task as finished

\texttt{\textgreater \ finish 2\\
  \textgreater \ ls\\
  1   Sample task 1\\
  2 f Sample task 2}
 
To finish an existing task, use the finish command followed by the serial number of the task to finish. 
Notice that for finished task, a ``f'' is shown between serial number and task description.

\item Remove task(s)

\texttt{\textgreater \ ls\\
    1 f Sample task 1\\
    2   Sample task 2\\
    3   Sample task 3\\
    \textgreater \ rm 1 2\\
    TaskManager: Do you really want to remove these tasks permanently? \textcolor{blue}{y}\\
    \textgreater \ ls\\
    3   Sample task 3}

To remove a existing task, use \texttt{rm} command followed by the serial number(s) of the task to remove. TaskManager will prompt for confirmation when removing tasks.

\item Exit from taskManager

\texttt{\textgreater \ exit}

To quit from taskManager, use \texttt{exit} command. 
All changes to the existing tasks will be automatically saved.
\end{enumerate}

\subsection{More commands}
\subsubsection{read, import and export}

TaskManager stores the tasks in an XML file which is by default located at
\texttt{\~{}/record.xml} on *nix, and \texttt{\%USERPROFILE\%\textbackslash record.xml} on Windows.

TaskManager also supports importing and exporting. This is done by \texttt{read}, \texttt{import} and \texttt{export} commands.

\begin{enumerate}
\item \texttt{read}

\texttt{read} command reads an xml file, list all the tasks it contains without affecting the current task list.

This is helpful when you only want to peek the content of an xml file without really importing it.

\texttt{\textgreater \ ls\\
    1   Sample task 2\\
    \textgreater \ read thisweek.xml \footnotemark \\
    1 f CS2103 midTerm Sep 29 06:30 - 07:30 pm MPSH 1B\\
    2 f CS3230 midTerm Oct 15 06:00 pm\\
    3 f CS3241 midTerm Oct 07 lecture\\
    4 f CS3244 midTerm Oct 04 lecture\\
    5 f ST2132 midTerm Oct 08 LT33 12:15 - 1:30 pm\\
    \textgreater \ ls
    1   Sample task 2}

\footnotetext{ The file name is not quoted. If the file name contains space, please quote it with a pair of quotation marks.}

\item \texttt{import}

\texttt{import} command is similar to \texttt{read} command. It reads the content of the xml file and appends all the tasks in it to current task list.

\texttt{\textgreater \ ls\\
    1   Sample task 2\\
    \textgreater \ import mytasks.xml\\
    1   Sample task 2 \footnotemark\\
    2 f CS2103 midTerm Sep 29 06:30 - 07:30 pm MPSH 1B\\
    3 f CS3230 midTerm Oct 15 06:00 pm\\
    4 f CS3241 midTerm Oct 07 lecture\\
    5 f CS3244 midTerm Oct 04 lecture\\
    6 f ST2132 midTerm Oct 08 LT33 12:15 - 1:30 pm\\}

\footnotetext{ Task 1 is still in task list. Importing tasks will not erase existing tasks.}

\item \texttt{export}

\texttt{export} command exports current task list to an xml file or html file.

\texttt{\textgreater \ export sampletasks.xml \footnotemark\\
  \textgreater \ exit\\
  \$ cat sampletasks.xml\\
    <taskList>\\
        <task>\\
            <serialNumber>1</serialNumber>\\
            <deadline>1288473083</deadline>\\
            <priority>0</priority>\\
            <description>Sample task 2</description>\\
            <group>default</group>\\
            <isFinished>0</isFinished>\\
        </task>\\
    </taskList>}

\footnotetext{ Currently \texttt{export} does not support environmental variables in path. E.g. export \~{}/abc.xml will not export the file to /home/username/.}
 
\texttt{export} can also be used to generate an html file which is more visually pleasant in your favourate browser.

\texttt{\textgreater \ export -html sampletasks.html\\
    \textgreater \ exit}

\begin{figure}[htbp]
  \centering
  \includegraphics[scale=0.73, viewport=100 0 600 380]{exporthtml}
  \caption{tasks exported as webpage \footnotemark}
\end{figure}

\footnotetext{ Page may not render correctly in IE 6 or its earlier versions.}
\end{enumerate}

\subsubsection{task}

\texttt{\textgreater \ task 1\\
  Number:  1       Deadline:   Sun Oct 31 05:11:23 2010\\
  Priority:  0         Status:   Doing\\
  Group:  default\\
  Details:\\
  Sample task 2 \footnotemark \\}

To show detail information of a task, use \texttt{task} command followed by serial number of the task.

\footnotetext{ Adding tasks with detailed information is covered in chapter 3. In this example, default values are shown.}

\subsubsection{pri}

\texttt{\textgreater \  pri 1 10\\
  \textgreater \ task 1\\
  Number:  1       Deadline:   Sun Oct 31 05:11:23 2010\\
  Priority:  10        Status:   Doing\\
  Group:  default\\
  Details:\\
  Sample task 2\\}

To change the priority of a task, use \texttt{pri} command followed by the serial number of a task and its new priority.
Priority is typically a number between -20 and 20. By default, the priority of a newly added task is 0.

\subsubsection{edit}

\texttt{\textgreater \ edit 1 -d "Sample task 3" -p 12 -t 1d\footnotemark \ -g SampleGroup -f yes\\
  \textgreater \ task 1\\
  Number:  1       Deadline:   Mon Nov  1 05:47:59 2010\\
  Priority:  12        Status:   Finished\\
  Group:  SampleGroup\\
  Details:\\
  Sample task 3\\}

To edit a task, \texttt{edit} command is used like this:
\texttt{edit \textless task serial number\textgreater \ -d \textless new description\textgreater \ -p \textless new priority\textgreater \ -t \textless new deadline\textgreater \ -g \textless new group\textgreater \ -f \textless finished or not\textgreater}

Only serial number is compulsory. Besides, to finish a task, \texttt{finish 1} is equivalent to \texttt{edit 1 -f yes}.

\footnotetext{ -t 1d means setting the deadline to be 1 day later. Time format that taskManager accepts is discussed in section 3.1}

\subsubsection{undo}
\texttt{\textgreater \ undo \footnotemark}

Undo the last command. 

\footnotetext{ The undo command has no effect on commands like ls, export, tui and undo.}

\subsubsection{redo}
\texttt{\textgreater \ redo}

Redo the last undo. If more than one undo command executed, redo the next latest one.

\subsection{Using Options}

Like \texttt{edit}, some of the commands come with options to support more functionality.
In this chapter they are introduced in great detail.

\subsubsection{add}

Use -t option to add a task with a deadline:
\texttt{\textgreater \ add "some task" -t 3d2h\\}

    taskManager support 3 types of time format: 
    Plus format:        \textless number:a\textgreater d \textless number:b\textgreaterh\textlessnumber:c\textgreaterm
                        Means a days b hours c minutes later. i.e. current time plus a day b hours c minutes.
                        None of them is compulsory
                        e.g. 3d2h means 3 days 2 hours from the moment the command is executed.
    By format:          b\textlessnumber:a\textgreaterw\textlessnumber:b\textgreaterd\textlessnumber:c\textgreaterh\textlessnumber:d\textgreaterm
                        Means before c hour d minutes on the b-th day of a-th week from now.
                        None of them is compulsory
                        e.g. b0w5d means by friday this week.
                             b2w3d8h means by wednesday 8:00am, 2 weeks later.
                             b1w means by start of next week, i.e. by the end of this week.
                             b2d means by the end of tomorrow.
                             b0d22h means by 10:00pm today.
                             b10d means by start of the 10-th day from now, i.e. in 10 days including today.

    Unix timestamp:     number of seconds since 1/1/1900 00:00:00

Use -p option to add a task with a priority
\texttt{\textgreater add "some important task" -p 20}

Use -g option to specify a group for a task
\texttt{\textgreater \ add "the task with group"\footnotemark \  -g SampleGroup}

\footnotetext{ If group name contains spaces, use a pair of quotation marks to quote it.}

Options are not compulsory. Different options can be used together. For example \texttt{add "some task" -p 10 -g "special task" -t 4d}.

\subsubsection{ls}

Use -s to sort the task list

    \textgreater \ ls -s "deadline priority"
    
    A more general format is:

    \textgreater \ ls -s "\textlesssearch keyword 1\textgreater \textlesssearch keyword 2\textgreater ..."

    The listed tasks will be sorted by keyword 1 then keyword 2 ...
    
    Available search keywords are: deadline priority serialnumber. Prefix of a keyword is also acceptable. e.g. \textgreaterls -s "p" will sort the list by priority
    e.g.

    \textgreater \ ls -s "p d"

    1   Sample task 1. This also has high priorty   10  Sun Oct 31 06:49:09 2010
    2   Sample task 2. This has high priority       10  Mon Nov  1 06:54:42 2010
    3   Sample task 3. This is the latest           0   Tue Nov  2 06:54:37 2010


Use -k to filter tasks with a keyword

    \textgreater \ ls -k *Sam?le*task

    ? means can be replaced with any character.
    * means can be replaced with any number of ?s. (including 0)
    keyword is case insensitive

    E.g.    "This is a sample with a important task" will match *Sam?le*task by letting first * to be "This is a ", ? to be 'p' and second "*" to be " with a important " 
            "samqleTask" will also match *sam?le*task by letting both * to be empty string and ? to be 'q' 

Use -f to show/hide finished tasks

    \textgreater \ ls -f yes

    Show only finished tasks

    \textgreater \ ls -f N

    Show only doing tasks

Use -g to show tasks of a specific group

    \textgreater \ ls -g SampleTask

    Show only tasks from SampleTask group

Notice: Different options can be used together.
        When more then one restrictive options are there, conjuction of these restrictions are used.
        e.g. ls -g SampleTask -f y will show finished tasks from SampleTask group. 

\subsubsection{rm}
Use -g option to remove a group of tasks:

\texttt{rm -g SampleTask} removes the entire SampleTask group.

\texttt{rm} can be used to remove a list of tasks as well. e.g. \texttt{rm 1 2 3} removes tasks 1, 2 and 3.

%% \subsubsection{edit}
%%     This is discussed in section 2.2.4.
%% \subsubsection{export}
%%     This is discussed in section 2.2.1.

Notice: Commands like finish, rm, export, etc. do not support all task-selective options like -g -k -f.
Executing these commands on a selected task set can be done with command piping, which is discussed in section 2.5.1.

\subsection{Text Based Interactive User Interface}
blah blah blah

\subsection{Advanced Usages}
\subsubsection{command piping}

    \textgreater \ ls | rm

TaskManager support command piping for most commands. Though it is a bit different from traditional linux/unix pipe.
Piping in taskManager is done with symbol '|'. When a few taskManager commands are piped together, they are executed one by one from left to right. 
If one command has output, it is then passed to the next command as input. Only output of the last command is shown.

Examples:
    
a) finish all tasks 

    \textgreater \ ls | finish

b) remove all finished tasks

    \textgreater \ ls -f yes | rm

c) import from a file and replace corrent task list

    \textgreater \ ls | rm | import newTasks.xml

d) import all CS2103 group tasks from a file

    \textgreater \ read newTasks.xml | ls -g CS2103 | add

e) export all CS2103 related tasks to a html file
    
    \textgreater \ ls -k *CS2103* | export -html cs2103tasks.html

f) show detail of CS2103 tasks, sort by priority

    \textgreater \ ls -g CS2103 | sort "pri" | task


\subsubsection{command mapping}

    \textgreater \ map "ls" "ls -f no"

TaskManager support custom command mapping/alias. General format of map is like this:

    \textgreater \ map "\textlessnew command\textgreater" "\textlessoriginal command\textgreater"
    \textgreater \ map "\textlesspattern of aliased command\textgreater" "\textlesspattern of original command\textgreater"

A simple mapping is like the previous example. This maps "ls" to "ls -f no", which means hide finished tasks when listing.
To have another command which shows all tasks, do this:

    \textgreater \ map "lsa" "ls"
    \textgreater \ map "ls" "ls -f no"

Notice: The order of mapping matters as commands are executed one by one. Reversing the order of these two mapping will not work.

More complex mapping makes use of \$ symbol. There are two kind of \$ symbols:
    \$0 Means all characters from surrent position
    \$1, \$2, \$3 .... Means one token

E.g.
    
    \textgreater \ map "tomorrow \$1" "add \$1 -t 1d"
    \textgreater \ tomorrow "Finish user guide"
    
    The latter command will be parsed as
    \textgreater \ add "Finish user guide" -t 1d
    and a new task will be added with the deadline to be 1 days later

    \textgreater \ map "do \$1 at \$2" "add \$1 -t \$2"
    \textgreater \ do "Laundry" at 4h
    
    The latter command will be parsed as
    \textgreater \ add "Laundry" -t 4h
    and a new task called Laundry will be added with the deadline to be 4hours later

    \textgreater \ map "ls \$0" "ls -f no \$0"
    \textgreater \ ls
    \textgreater \ ls -g cs2103
    
    The second command will be parsed as
    \textgreater \ ls -f no 
    and will list out all unfinished tasks.
    The third command will be parsed as
    \textgreater \ ls -f no -g cs2103
    and will list out all unfinished cs2103 tasks.

Notice: TUI uses ls to retrieve tasks. Mapping ls to something else will affect behaviour of TUI.
    
\subsubsection{taskManager script}

Task manager commands can be saved in a single script file and be executed using run command.

    \$ cat tmscript 
    ls
    map "ls" "ls -f no"
    ls

    \$ ./taskManager

     Task Manager V 0.1 
     exit\textlessenter\textgreater to quit. help\textlessenter\textgreater for more instructions
    =====================================================================
    \textgreater \ run tmscript

    1 f Sample task 1. This also has high priorty
    2   Sample task 2. This has high priority
    3   Sample task 3. This is the latest


    2   Sample task 2. This has high priority
    3   Sample task 3. This is the latest

Notice: The first 3 tasks are the result of the first ls in the script.
        The last 2 tasks are the result of the second ls in the script.
        Because "ls" is mapped to "ls -f no", finished tasks are not shown by the second ls.
        TaskManager scripts are pure text files. ".txt" suffix is omitted here.
        In windows, suffix is hidden by default. To show it, check corresponding checkbox in tools -\textgreater folder options. 
        Windows may not know how to open file with no suffix.

\subsubsection{startup script}

By default, taskManager executes a special script everytime when it is started. 
This script is located at:

    \~{}/.tmrc                 \textless linux/mac os\textgreater
    \%USERPROFILE\%\textbackslash tmrc.txt  \textless Windows\textgreater

This file can be editted to include customized setting.

E.g.

    a) To switch to the interactive user interface by default, add this line into tmrc
    tui

    b) To save a backup file when taskManager is started, add this line into tmrc
    export /tmp/backupTasklist.xml

    c) To show tasks when taskManager is started, add this line into tmrc
    ls

    d) To remove finished tasks when taskManager is started, add this line into tmrc
    ls -f yes | rm

    e) To run a script with all self-defined mappings when taskManager is started, add this line into tmrc
    run /home/myusername/mymappings

5.5 Talk to taskManager

For all inputs that cannot be recognized by taskManager as a command, it will be treated as natural language sentense. 
TaskManager will try its best to recognize it and give correct response.

E.g.

    \textgreater \ what do I do today?

All tasks due today will be listed out.    

\subsection{Compilation and Installation}
\subsubsection{Microsoft Windows}

On Windows, taskManager can be built with Visual Studio in the following steps:
\begin{enumerate}
\item Start Visual Studio with ``C++ Development Settings''.
\item Create a win32 console project.
\item Drag all source files into the solution folder. Files should be automatically categorized into header files and source files.
\item Edit project configuration, under general, set character set to be multi-byte characters.
\item Edit project configuration, under linker, add pdcurses.lib to additional libraries and add the folder containing pdcurses.lib to library search directories.
\item Copy pdcurses.dll to \%WINDIR\%\textbackslash system32\textbackslash or the directory your executable will be generated.
\item Build the solution.
\end{enumerate}

To build taskManager with TUI, pdcurses library is needed.
It is free and can be downloaded here:
\url{http://sourceforge.net/projects/pdcurses/files/}

Edit the project property to include the pdcurses.lib and in the linker options.

\subsubsection{Unix-like Operating Systems}

In Unix-like operating systems like GNU Linux and Mac OS X, start a shell, change directory to taskManager's source folder and type:

\texttt{\$ make\\
  \$ sudo make install}

da ``make install'' is optional. It just makes taskManager available system wide, which copys the executable and man pages to cooresponding directories.

The text UI is built by default, which requires ncurses library.
It ships with most Linux distributions and Mac OS. If not, it can be installed with the package manager (apt-get, yum, pacman on various Linux distributions and port on Mac).

\section{Developer Guide}
ahah

\end{document}
